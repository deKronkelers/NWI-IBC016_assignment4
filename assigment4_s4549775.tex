\documentclass[12pt]{article}

\usepackage{amsmath}
\usepackage{amssymb}
\usepackage{enumerate}

\setlength\parindent{0em}
\setlength\parskip{1em}

\title {Assignment 4}

\author {Hendrik Werner s4549775}

\begin{document}
\maketitle

\section*{10}
I assume that a committee is like a set meaning that the order does not matter, which is why I use $C$ instead of $P$. $P$ would give a number that is too high because it counts different constellations of the same committee as different committees.

$\sum_{i = 4}^{6}(C(10, 6 - i)C(15, i))\\
= C(10, 2) * C(15, 4) + C(10, 1) * C(15, 5) + C(10, 0) * C(15, 6)\\
= 45 * 1365 + 10 * 3003 + 1 * 5005 = 96460$

\section*{11}
In license plates the order does matter so we need to use $P$.

$P(26, 3)P(10, 3) = 15 600 * 720 = 11 232 000$

In combinatorial terms this means choosing a permutation of 3 letters, than choosing a permutation of 3 digits.

\section*{12}
We need to enumerate all possible combinations of medals, compute the number of possibilities for each and sum them.

G = gold\\
S = silver\\
B = bronce

\begin{description}
	\item[6 medals awarded]
	GGGGGG
	GGBBBB
	GSSSSS
	GSBBBB

	\item[5 medals awarded]
	GGGGG
	GGBBB
	GSSSS
	GSBBB

	\item[4 medals awarded]
	GGGG
	GGBB
	GSSS
	GSBB

	\item[3 medals awarded]
	GGG
	GGS
	GSS
	GSB
\end{description}

\section*{13}
$\sum_{i = 0}^{200}$$(\binom{200}{i} 2x^{200 - i} * (-3)y^i)\\
= \binom{200}{0} 2x^{200 - 0} * (-3)y^0 + \binom{200}{1} 2x^{200 - 1} * (-3)y^1 + \dots + \binom{200}{200} 2x^{200 - 200} * (-3)y^{200}$

The term which contains $x^{101}y^{99}$ is $\binom{200}{99} 2x^{101} * (-3)y^{99}$ so the coefficient of $x^{101}y^{99}$ is $\binom{200}{99} * 2^{101} * (-3)^{99}$.

\section*{14}
\begin{enumerate}[a]
	\item %a
	$\binom{n}{r}\binom{r}{k}$ means that you choose $r$ elements from a set of $n$ elements and then choose $k$ elements from the set of $r$ elements you chose first.

	$\binom{n}{k}\binom{n - k}{r - k}$ means that you choose $k$ elements out of $n$ elements and choose $r - k$ from the remaining $n - k$ elements.

	It is arbitrary whether you say "I choose $k$ elements out of $n$." or "I leave $n - k$ out of $n$ elements over." so when there are the same amount of elements left over than both methods are equal.

	$n - r = n - k - (r - k)\\
	n - r = n - k - r + k\\
	n - r = n - r$ \checkmark

	\item % b
	We want to prove: $C(n, r)C(r, k) = C(n, k)C(n - k, r - k)$

	$C(n, r)C(r, k) = C(n, k)C(n - k, r - k)\\
	C(n, r)C(r, k) = C(n, k)C(n - k, n - k -(r - k))\\
	C(n, r)C(r, k) = C(n, k)C(n - k, n - k - r + k)\\
	C(n, r)C(r, k) = C(n, k)C(n - k, n - r)\\
	C(n, r)C(r, k) = C(n, k) \dfrac{(n - k)!}{(n - r)!(n - k - (n - r))!}\\
	C(n, r)C(r, k) = C(n, k) \dfrac{(n - k)!}{(n - r)!(n - k - n + r))!}\\
	C(n, r)C(r, k) = C(n, k) \dfrac{(n - k)!}{(n - r)!(r - k))!}\\
	C(n, r)C(r, k) = \dfrac{n!}{k!(n - k)!} * \dfrac{(n - k)!}{(n - r)!(r - k))!}\\
	C(n, r)C(r, k) = \dfrac{n!(n - k)!}{k!(n - k)!(n - r)!(r - k))!}\\
	C(n, r)C(r, k) = \dfrac{n!}{k!(n - r)!(r - k))!}\\
	\dfrac{n!}{r!(n - r)!} C(r, k) = \dfrac{n!}{k!(n - r)!(r - k))!}\\
	\dfrac{n!}{r!(n - r)!} \dfrac{r!}{k!(r - k)!} = \dfrac{n!}{k!(n - r)!(r - k))!}\\
	\dfrac{n!r!}{r!(n - r)!k!(r - k)!} = \dfrac{n!}{k!(n - r)!(r - k))!}\\
	\dfrac{n!}{(n - r)!k!(r - k)!} = \dfrac{n!}{k!(n - r)!(r - k))!}\\
	\dfrac{n!}{k!(n - r)!(r - k)!} = \dfrac{n!}{k!(n - r)!(r - k))!}$ \checkmark
\end{enumerate}

\end{document}
